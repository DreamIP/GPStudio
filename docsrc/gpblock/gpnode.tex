\documentclass[10pt,a4paper]{article}
\usepackage[utf8]{inputenc}
\usepackage[english]{babel}
\usepackage[T1]{fontenc}
\usepackage{amsmath}
\usepackage{amsfonts}
\usepackage{amssymb}
\usepackage{lmodern}
\usepackage{fancyvrb}
\usepackage{tikz}
\usepackage{verbatim}
\usepackage{indentfirst}

\newcommand{\version}{\IfFileExists{../../version.txt}
{\input{../../version.txt}}
{\input{../../../version.txt}}
}

\newcommand{\command}[1]{%
\indent \fcolorbox{black}{white}{%
   \begin{minipage}{\dimexpr\textwidth-\parindent\relax}%
      #1
   \end{minipage}%
}
}

\newsavebox{\FVerbBox}
\newenvironment{sample}
{\par \vspace{0.2cm} \begin{lrbox}{\FVerbBox}
\begin{minipage}{\dimexpr\textwidth-\parindent\relax}}
{\end{minipage}
\end{lrbox}
\fcolorbox{black}{lightgray}{\usebox{\FVerbBox}}
\vspace{0.2cm}}

\newenvironment{sampletitle}
{\vspace{0.2cm} \noindent\textbf{Example} :
\begin{sample}}
{\end{sample}}

\newcommand{\samplecomment}[1]{%

\textit{#1}
}

\newcommand{\seealso}[1]{\vspace{0.2cm} \noindent\textbf{See also} :\par #1}

% tikz
\usetikzlibrary{calc}
\usetikzlibrary{arrows}
\usetikzlibrary{shadows}

\tikzset{block/.style={draw, text centered, fill=gray!10,drop shadow}}
\tikzset{connect/.style={draw, line width=1 pt}}

\author{Sebastien CAUX}
\title{gpnode reference documentation \version}

\begin{document}
\maketitle
\section{Introduction}
gpnode is a command line tool that permits to create and manage a node project in GPStudio toolchain.

A node in GPStudio is a physical node, it can be a smart camera or a sensor.

\section{Use}
gpnode always takes the project in the current directory, so you only can have one project per directory. A node project file have the `.node' extension.

At the beginning, you need to create a project with the \emph{newproject} command. After that, you can use all the commands set on this project.

Please read the tutorial `GPStudio command line quick start' to learn how to use this tool.

Under linux, you have a completion script to help you writing commands.

\section{Commands}
\subsection{project}
\subsubsection{newproject}
\command{\textbf{gpnode} \textbf{newproject} -n <project-name>}

Create a project file inside the current directory named `\texttt{<project-name>}.node'.

\paramcommand{
-n & project name without space & project1 \\ 
}

\begin{sampletitle}
> \textbf{gpnode} \textbf{newproject} -n project1
\end{sampletitle}
\samplecomment{Create a new project named \texttt{project1}. After that, you have a file project \texttt{project1.node} in the current directory.}

\subsubsection{setboard}
\command{\textbf{gpnode} \textbf{setboard} -n <board-name>}

You need to specify a for the used board board before setting up any io in the project.

\paramcommand{
-n & board name support, contained in library & dreamcam\_c3 \\ 
}

\begin{sampletitle}
> \textbf{gpnode} \textbf{setboard} -n dreamcam\_c3
\end{sampletitle}
\samplecomment{Your project is now based on the dreamcam platform. You can now use all the image sensors and communication for this platform.}

\seealso{\textbf{gplib listboards} and \textbf{gpnode showboard}}

\subsubsection{showboard}
\command{\textbf{gpnode} \textbf{showboard}}

Prints the name of the board specified in the current project.

\begin{sampletitle}
> \textbf{gpnode} \textbf{showboard}
\begin{Verbatim}
dreamcam_c3
\end{Verbatim}
\end{sampletitle}

\seealso{\textbf{gpnode setboard}}

\subsubsection{generate}
\command{\textbf{gpnode} \textbf{generate} [-o <dir>]}

Generates all the files needed for the specified toolchain and Makefile. After that, you just need to call `make compile' in the directory \texttt{build/} to compile the project with specific tools needed by the node.

\paramcommand{
-o & output directory & build \\ 
}

\begin{sampletitle}
> \textbf{gpnode} \textbf{generate} -o build/\\
> cd build\\
> make compile
\end{sampletitle}
\samplecomment{Generates the project in the subdirectory \texttt{build/} with a Makefile. In this directory, you can execute make compile to call the compiler for the specified platform.}

\subsection{io}
\subsubsection{addio}
\command{\textbf{gpnode} \textbf{addio} -n <io-name>}

Adds IP support in the project to manage \texttt{<io-name>}. \texttt{<io-name>} must be define in the board file definition.

\paramcommand{
-n & io name specified in the support board package & mt9 \\ 
}

\begin{sampletitle}
> \textbf{gpnode} \textbf{addio} -n mt9\\
> \textbf{gpnode} \textbf{showblock}
\begin{Verbatim}
blocks :
  + mt9 [mt9]
\end{Verbatim}
\end{sampletitle}
\samplecomment{Adds the support for mt9 image sensor. You have a block named mt9 in the project.}

\seealso{\textbf{gpnode showio}, \textbf{gpnode delio} and \textbf{gpnode listavailableio}}

\subsubsection{delio}
\command{\textbf{gpnode} \textbf{delio} -n <io-name>}

Removes io support named \texttt{<io-name>}.

\paramcommand{
-n & io name almost added & mt9 \\ 
}

\begin{sampletitle}
> \textbf{gpnode} \textbf{delio} -n mt9
\end{sampletitle}
\samplecomment{Removes the `mt9' block with its support and all the flow connection from it.}

\seealso{\textbf{gpnode showio} and \textbf{gpnode addio}}

\subsubsection{showio}
\command{\textbf{gpnode} \textbf{showio}}

Prints the list of all the io support in the current project. The output format is : + \texttt{<block-name>} [\texttt{<block-driver>}]

\begin{sampletitle}
> \textbf{gpnode} \textbf{showio}
\begin{Verbatim}
ios :
  + mt9 [mt9]
  + usb [usb_cypress_CY7C68014A]
\end{Verbatim}
\end{sampletitle}

\subsubsection{listavailableio}
\command{\textbf{gpnode} \textbf{listavailableio}}

Prints the list of all available ios for the platform that you have specified before.

\begin{sampletitle}
> \textbf{gpnode listavailableio}
\begin{Verbatim}
led mt9 e2v ethernet usb
\end{Verbatim}
\end{sampletitle}

\subsection{process}
\subsubsection{addprocess}
\command{\textbf{gpnode} \textbf{addprocess} -n \emph{<process-name>} -d \emph{<driver-name>}}

Adds a process named \texttt{<process-name>} as an instance of \texttt{<driver-name>} IP in the library or the project dir.

\paramcommand{
-n & name of the process instance & process1 \\ 
\hline
-d & driver name to instantiate. Could be defined in library or locally in the project with gpblock. & gradianthw \newline myproc/myproc \\ 
}

\begin{sampletitle}
> \textbf{gpnode} \textbf{addprocess} -n process1 -d gradienthw\\
> \textbf{gpnode} \textbf{showprocess}
\begin{Verbatim}
process :
  + process1 [gradienthw]
\end{Verbatim}
\end{sampletitle}
\samplecomment{Adds a process named process1 based on a process declared in library gradienthw.}

\seealso{\textbf{gpnode delprocess} and \textbf{gplib listprocess}}

\subsubsection{delprocess}
\command{\textbf{gpnode} \textbf{delprocess} -n <process-name>}

Removes process \texttt{<process-name>} and all the connections to or from it.

\paramcommand{
-n & process name almost added & process1 \\ 
}

\begin{sampletitle}
> \textbf{gpnode} \textbf{delprocess} -n process1
\end{sampletitle}
\samplecomment{Removes `process1'.}

\seealso{\textbf{gpnode addprocess}}

\subsubsection{showprocess}
\command{\textbf{gpnode} \textbf{showprocess}}

Prints the list of processes in the current project. The output format is : + \texttt{<block-name>} [\texttt{<block-driver>}]

\begin{sampletitle}
> \textbf{gpnode} \textbf{showprocess}
\begin{Verbatim}
process :
  + process1 [gradienthw]
  + process2 [lbp]
\end{Verbatim}
\end{sampletitle}

\subsubsection{showblock}
\command{\textbf{gpnode} \textbf{showblock} $\big[$-n \emph{<process-name>} [-d \emph{<driver-name>}]$\big]$}

If -n option is not set, it prints the list of processes and io in the current project. The output format is : + \texttt{<block-name>} [\texttt{<block-type>} - \texttt{<block-driver>}]

\paramcommand{
-n & name of the process instance to show and only this one & process1 \\ 
\hline
-f & filter & flows \newline params \newline clocks \newline resets \\ 
}

\begin{sampletitle}
> \textbf{gpnode} \textbf{showblock}
\begin{Verbatim}
blocks :
  + led [io - leds]
  + mt9 [io - mt9]
  + usb [iocom - usbcypressCY7C68014A]
  + process1 [process - gradienthw]
  + conv [process - conv]
  + lbp [process - lbp]
\end{Verbatim}
\end{sampletitle}

If -n option is set, it prints the list of params, clocks, resets and flows of the block \texttt{<block-name>}.

\begin{sampletitle}
> \textbf{gpnode showblock} -n usb
\begin{Verbatim}
flows :
       -------------        
  in0  |           |  out0  
------>|           |------->
  in1  |           |  out1  
------>|           |------->
  in2  |    usb    |        
------>|           |        
  in3  |           |        
------>|           |        
       -------------
params :
  + generic IN0NBWORDS type: int value: 32768
  + generic IN1NBWORDS type: int value: 32768
  + generic IN2NBWORDS type: int value: 32768
  + generic IN3NBWORDS type: int value: 32768
  + generic OUT0NBWORDS type: int value: 1024
  + generic OUT1NBWORDS type: int value: 1024
  + register status regaddr: 0 propertymap: 
  + register flowin0 regaddr: 1 propertymap: -
  + register flowin1 regaddr: 2 propertymap: -
  + register flowin2 regaddr: 3 propertymap: -
  + register flowin3 regaddr: 4 propertymap: -
clocks :
  + clkproc  Hz in
  + clkusb 48 MHz out
resets :
  + reset resetn out
\end{Verbatim}
\end{sampletitle}

If -f is specified with -n, the command shows only the list of \texttt{<filter>}.

\begin{sampletitle}
> \textbf{gpnode showblock} -n usb -f params
\begin{Verbatim}
params :
  + generic IN0NBWORDS type: int value: 32768
  + generic IN1NBWORDS type: int value: 32768
  + generic IN2NBWORDS type: int value: 32768
  + generic IN3NBWORDS type: int value: 32768
  + generic OUT0NBWORDS type: int value: 1024
  + generic OUT1NBWORDS type: int value: 1024
  + register status regaddr: 0 propertymap: 
  + register flowin0 regaddr: 1 propertymap: -
  + register flowin1 regaddr: 2 propertymap: -
  + register flowin2 regaddr: 3 propertymap: -
  + register flowin3 regaddr: 4 propertymap: -
\end{Verbatim}
\end{sampletitle}
\samplecomment{Prints only the list of params}

\subsection{block attributes}
\subsubsection{renameblock}
\command{\textbf{gpnode} \textbf{renameblock} -n <block-name> -v <new-name>}

Renames block the block named \texttt{<block-name>} with the name \texttt{<new-name>}. Please notify that io block could not be renamed.

\paramcommand{
-n & name of the process instance to rename & process1 \\ 
\hline
-v & new block instance name & convolve \\ 
}

\begin{sampletitle}
> \textbf{gpnode} \textbf{renameblock} -n process1 -v first\_gradient
\end{sampletitle}
\samplecomment{Renames the block named `process1' with the name `first\_gradient'.}

\subsubsection{setproperty}
\command{\textbf{gpnode} \textbf{setproperty} -n <property-name> -v <default-value>}

Defines a default value \texttt{<default-value>} to the property \texttt{<property-name>}.

\paramcommand{
-n & name of the property composed by the name of the block, a (dot) and the name of the property. A property could be a subproperty of another, in this case, name of the block, (dot), name of the parent property, (dot) and name of the child property. & process1.prop \newline process1.prop.prop2 \\ 
\hline
-v & value to give to the property & 50 \newline true \\ 
}

\begin{sampletitle}
> \textbf{gpnode} \textbf{setproperty} -n mt9.roi1.w -v 1280 \\
> \textbf{gpnode} \textbf{setproperty} -n mt9.enable -v true
\end{sampletitle}
\samplecomment{When you launch the camera, you have an image from mt9 with 1280 pixels of width. The mt9 block is enabled at the beginning.}

\subsubsection{setparam}
\command{\textbf{gpnode} \textbf{setparam} -n <param-name> -v <value>}

Sets the value \texttt{<value>} to the param \texttt{<param-name>}. If <param-name> is a constant parameter, it sets the value. If it is a register, it sets the default value.

\paramcommand{
-n & name of the param composed by the name of the block, a (dot) and the name of the param & process1.LINEW \\ 
\hline
-v & value to give to the param & 1280 \newline ON \\ 
}

\begin{sampletitle}
> \textbf{gpnode} \textbf{setparam} -n usb.IN0\_NBWORDS -v 2048
\end{sampletitle}
\samplecomment{Redefines the parameter IN0\_NBWORDS of usb block to 2048 blocks. It is the size of the fifo for in0 input flow.}

\subsubsection{setclock}
\command{\textbf{gpnode} \textbf{setclock} -n <clock-name> -v <frequency>}

Defines the clock frequency \texttt{<frequency>} to the clock \texttt{<clock-name>}.

\paramcommand{
-n & name of the clock composed by the name of the block, a (dot) and the name of the clock. & process1.clockimg \\ 
\hline
-v & frequency to give to the clock. It is possible to use multiplier suffix like 'G', 'M' or 'k'. & 0.25G \newline 62M \newline 5.5k \\ 
}

\begin{sampletitle}
> \textbf{gpnode} \textbf{setclock} -n mt9.clk\_img -v 10M
\end{sampletitle}
\samplecomment{Redefines the input pixel clock of mt9 io block.}

\subsubsection{setflowsize}
\command{\textbf{gpnode} \textbf{setflowsize} -n <flow-name> -v <flow-size>}

Redefines the flow size \texttt{<flow-size>} to the flow \texttt{<flow-name>}.

\paramcommand{
-n & name of the flow composed by the name of the block, a (dot) and the name of the flow & process1.in \\ 
\hline
-v & size in bits & 8 \\ 
}

\begin{sampletitle}
> \textbf{gpnode} \textbf{setflowsize} -n usb.in0 -v 16
\end{sampletitle}
\samplecomment{Redefines the width of input flow in0 of usb block to 16 bits.}

\subsection{flow interconnect}
\subsubsection{connect}
\command{\textbf{gpnode} \textbf{connect} -f <flow-out> -t <flow-in> [-s <bit-shift>]}

Adds a flow connection between a flow out \texttt{<flow-out>} (ex: mt9.out) to a flow in \texttt{<flow-in>}. Option -s could be used in case your flow don't have the same data width. You can choose between `msb' or `lsb' connection.

\paramcommand{
-f & name of the flow source composed by the name of the block, a (dot) and the name of the flow & process1.out \\ 
\hline
-t & name of the flow in & process1.in \\ 
\hline
-s & order & msb \newline lsb \\ 
}

\begin{sampletitle}
> \textbf{gpnode} \textbf{connect} -f mt9.out -t usb.in0
\end{sampletitle}
\samplecomment{Connect the output `out' of mt9 block to the input `in0' of usb to have a direct connection between the image sensor and usb communication.}

\seealso{\textbf{gpnode unconnect} and \textbf{gpnode showconnect}}

\subsubsection{unconnect}
\command{\textbf{gpnode} \textbf{unconnect} -f <flow-out> -t <flow-in>}

Remove a flow connection between a flow out \texttt{<flow-out>} (ex: mt9.out) to a flow in \texttt{<flow-in>}.

\paramcommand{
-f & name of the flow out & process1.out \\ 
\hline
-t & name of the flow in & process1.in \\ 
}

\begin{sampletitle}
> \textbf{gpnode} \textbf{unconnect} -f mt9.out -t usb.in0
\end{sampletitle}
\samplecomment{Remove the direct connection between the image sensor and usb communication.}

\seealso{\textbf{gpnode connect} and \textbf{gpnode showconnect}}

\subsubsection{showconnect}
\command{\textbf{gpnode} \textbf{showconnect}}

Print the list of flow connections in the current project.

\begin{sampletitle}
> \textbf{gpnode} \textbf{showconnect}
\begin{Verbatim}
connects :
  + mt9.out -> usb.in0 (msb)
  + process1.magnitude -> usb.in0 (lsb)
  + mt9.out -> conv.in (msb)
  + conv.out -> usb.in1 (msb)
  + mt9.out -> process1.in (lsb)
  + mt9.out -> lbp.in (msb)
  + lbp.out -> usb.in1 (msb)
\end{Verbatim}
\end{sampletitle}

\seealso{\textbf{gpnode connect} and \textbf{gpnode unconnect}}

\subsection{clock interconnect}
\subsubsection{setclockdomain}
\command{\textbf{gpnode} \textbf{setclockdomain} -n <domain-name> -v <frequency>}

Define a clock frequency \texttt{<frequency>} the the clock domain \texttt{<domain-name>}.

\paramcommand{
-n & name of the clock domain & clk\_proc \\ 
\hline
-v & frequency to give to the clock domain. It is possible to use multiplier suffix like 'G', 'M' or 'k'. & 0.25G \newline 62M \newline 5.5k \\ 
}

\begin{sampletitle}
> \textbf{gpnode} \textbf{setclockdomain} -n clk\_proc -v 50M
\end{sampletitle}
\samplecomment{Set the main clock domain to 50MHz.}

\seealso{\textbf{gpnode showclockdomain}}

\subsubsection{showclockdomain}
\command{\textbf{gpnode} \textbf{showclockdomain}}

Print the list of clock domains in the current project.

\begin{sampletitle}
> \textbf{gpnode} \textbf{showclockdomain}
\begin{Verbatim}
domains :
  + clk_proc = 48 MHz
\end{Verbatim}
\end{sampletitle}

\seealso{\textbf{gpnode setclockdomain}}

\end{document}
