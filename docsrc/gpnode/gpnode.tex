\documentclass[10pt,a4paper]{article}
\usepackage[utf8]{inputenc}
\usepackage[english]{babel}
\usepackage[T1]{fontenc}
\usepackage{amsmath}
\usepackage{amsfonts}
\usepackage{amssymb}
\usepackage{lmodern}
\usepackage{fancyvrb}
\usepackage{tikz}

\newcommand{\version}{\IfFileExists{../../version.txt}
{\input{../../version.txt}}
{\input{../../../version.txt}}
}

\newcommand{\command}[1]{%
\indent \fcolorbox{black}{white}{%
   \begin{minipage}{\dimexpr\textwidth-\parindent\relax}%
      #1
   \end{minipage}%
}
}

\newsavebox{\FVerbBox}
\newenvironment{sample}
{\par \vspace{0.2cm} \begin{lrbox}{\FVerbBox}
\begin{minipage}{\dimexpr\textwidth-\parindent\relax}}
{\end{minipage}
\end{lrbox}
\fcolorbox{black}{lightgray}{\usebox{\FVerbBox}}
\vspace{0.2cm}}

\newenvironment{sampletitle}
{\vspace{0.2cm} \noindent\textbf{Example} :
\begin{sample}}
{\end{sample}}

\newcommand{\samplecomment}[1]{%

\textit{#1}
}

\newcommand{\seealso}[1]{\vspace{0.2cm} \noindent\textbf{See also} :\par #1}

% tikz
\usetikzlibrary{calc}
\usetikzlibrary{arrows}
\usetikzlibrary{shadows}

\tikzset{block/.style={draw, text centered, fill=gray!10,drop shadow}}
\tikzset{connect/.style={draw, line width=1 pt}}

\author{Sebastien CAUX}
\title{gpnode doc \version}

\begin{document}
\maketitle
\section{Introduction}
gpnode is a command line tool that permits to create and manage a node project in GPStudio toolchain.

A node in GPStudio is a physical node, it can be a smart camera or a sensor.

\section{Use}
gpnode always takes the project in the current directory, so you only can have one project per directory. A node project file have the '.node' extension.

At the very beginning, you need to create a project with the \emph{newproject} command. After that, you can use all the commands set on this project.

Please read the tutorial 'GPStudio command line quick start' to learn how to use this tool.

Under linux, you have a completion script to help you to writing commands.

\section{Commands}
\subsection{project}
\subsubsection{newproject}
\command{\textbf{gpnode} \textbf{newproject} -n <project-name>}

Create a project file inside the current directory named '\emph{<project-name>}.node'.

\sampletitle{> \textbf{gpnode} \textbf{newproject} -n project1}
\samplecomment{Create a new project named project1. After that, you may have a file project project1.node in the current directory.}

\subsubsection{setboard}
\command{\textbf{gpnode} \textbf{setboard} -n <board-name>}

Specify the name of the used board for the node. You need to specify a board before setting up any io in the project.

\sampletitle{> \textbf{gpnode} \textbf{setboard} -n dreamcam\_c3}
\samplecomment{Your project is now base on the dreamcam platform. You can now use all the image sensors and communication for this platform.}

\subsubsection{showboard}
\command{\textbf{gpnode} \textbf{showboard}}

Print the name of the board specified in the current project.

\sampletitle{> \textbf{gpnode} \textbf{showboard}\\
dreamcam\_c3}

\subsubsection{generate}
\command{\textbf{gpnode} \textbf{generate} [-o <dir>]}

Generate all files needed for the specified toolchain and Makefile. After that, you just need to call 'make compile' to compile the project with specific tools needed by the node.

\sampletitle{> \textbf{gpnode} \textbf{generate} -o build/\\
> cd build\\
> make compile}
\samplecomment{Build the project in the subdirectory build/ with a Makefile. In this directory, you can execute make compile to call the compiler for the specified platform.}

\subsection{io}
\subsubsection{addio}
\command{\textbf{gpnode} \textbf{addio} -n <io-name>}

Add IP support in the project to manage \emph{<io-name>}. \emph{<io-name>} must be define in the board file definition.

\sampletitle{> \textbf{gpnode} \textbf{addio} -n mt9\\
> \textbf{gpnode} \textbf{showblock}\\blocks :\\ + mt9 [mt9]}
\samplecomment{Add the support for mt9 image sensor. You will have a block named mt9 in the project.}

\subsubsection{delio}
\command{\textbf{gpnode} \textbf{delio} -n <io-name>}

Remove io support named \emph{<io-name>}.

\sampletitle{> \textbf{gpnode} \textbf{delio} -n mt9}
\samplecomment{Remove the mt9 support and the block mt9 with all the flow connection from it.}

\subsubsection{showio}
\command{\textbf{gpnode} \textbf{showio}}

Print the list of all the io support in the current project. The output format is : + <block-name> [<block-driver>]

\sampletitle{> \textbf{gpnode} \textbf{showio}\\ios :\\ + mt9 [mt9]\\ + usb [usb\_cypress\_CY7C68014A]}

\subsection{process}
\subsubsection{addprocess}
\command{\textbf{gpnode} \textbf{addprocess} -n \emph{<process-name>} -d \emph{<driver-name>}}

Add a process named \emph{<process-name>} as an instance of \emph{<driver-name>} IP in the library or the project dir.

\sampletitle{> \textbf{gpnode} \textbf{addprocess} -n process1 -d gradienthw\\
> \textbf{gpnode} \textbf{showprocess}\\process :\\ + process1 [gradienthw]}
\samplecomment{Add a process named process1 based on a process declared in library gradienthw.}

\subsubsection{delprocess}
\command{\textbf{gpnode} \textbf{delprocess} -n <process-name>}

Remove process \emph{<process-name>} and all the connections to or from it.

\sampletitle{> \textbf{gpnode} \textbf{addprocess} -n process1 -d gradienthw}
\samplecomment{Remove process1.}

\subsubsection{showprocess}
\command{\textbf{gpnode} \textbf{showprocess}}

Print the list of process in the current project. The output format is : + <block-name> [<block-driver>]

\sampletitle{> \textbf{gpnode} \textbf{showprocess}\\process :\\ + process1 [gradienthw]\\ + process2 [gradienthw]}

\subsubsection{showblock}
\command{\textbf{gpnode} \textbf{showblock}}

Print the list of process and io in the current project. The output format is : + <block-name> [<block-type> - <block-driver>]

\sampletitle{\textbf{gpnode} \textbf{showblock}\\blocks :\\
  + led [io - leds]\\
  + mt9 [io - mt9]\\
  + usb [iocom - usb\_cypress\_CY7C68014A]\\
  + process1 [process - gradienthw]\\
  + conv [process - conv]\\
  + lbp [process - lbp]}

\subsection{block attributes}
\subsubsection{renameblock}
\command{\textbf{gpnode} \textbf{renameblock} -n <block-name> -v <new-name>}

Rename a process block.

\sampletitle{> \textbf{gpnode} \textbf{renameblock} -n process1 -v first\_gradient}
\samplecomment{Rename the block named 'process1' with the name 'first\_gradient'.}

\subsubsection{setproperty}
\command{\textbf{gpnode} \textbf{setproperty} -n <property-name> -v <default-value>}

Define a default value \emph{<default-value>} to the property \emph{<property-name>}.

\sampletitle{> \textbf{gpnode} \textbf{setproperty} -n mt9.roi1.w -v 1280}
\samplecomment{When you launch the camera, you will have image from mt9 with a width of 1280 pixels.}

\subsubsection{setparam}
\command{\textbf{gpnode} \textbf{setparam} -n <param-name> -v <value>}

Set the value \emph{<value>} to the param \emph{<param-name>}.

\sampletitle{> \textbf{gpnode} \textbf{setparam} -n usb.IN0\_NBWORDS -v 2048}
\samplecomment{Redefine the parameter IN0\_NBWORDS of usb block to 2048 blocks. It's the size of the fifo for in0 input flow.}

\subsubsection{setclock}
\command{\textbf{gpnode} \textbf{setclock} -n <clock-name> -v <frequency>}

Define the clock frequency \emph{<frequency>} to the clock \emph{<clock-name>}.

\sampletitle{> \textbf{gpnode} \textbf{setclock} -n mt9.clk\_img -v 10M}
\samplecomment{Redefine the input pixel clock of mt9 io block.}

\subsubsection{setflowsize}
\command{\textbf{gpnode} \textbf{setflowsize} -n <flow-name> -v <flow-size>}

Redefine the flow size \emph{<flow-size>} to the flow \emph{<flow-name>}.

\sampletitle{> \textbf{gpnode} \textbf{setflowsize} -n usb.in0 -v 16}
\samplecomment{Redefine the width of input flow in0 of usb block to 16 bits.}

\subsection{flow interconnect}
\subsubsection{connect}
\command{\textbf{gpnode} \textbf{connect} -f <flow-out> -t <flow-in> [-s <bit-shift>]}

Add a flow connection between a flow out \emph{<flow-out>} (ex: mt9.out) to a flow in \emph{<flow-in>} with a bit shift \emph{<bit-shift>} ("lsb" or "msb").

\sampletitle{> \textbf{gpnode} \textbf{connect} -f mt9.out -t usb.in0}
\samplecomment{Connect the output 'out' of mt9 block to the input 'in0' of usb to have a direct connection between the image sensor and usb communication.}

\subsubsection{unconnect}
\command{\textbf{gpnode} \textbf{unconnect} -f <flow-out> -t <flow-in>}

Remove a flow connection between a flow out \emph{<flow-out>} (ex: mt9.out) to a flow in \emph{<flow-in>}.

\sampletitle{> \textbf{gpnode} \textbf{unconnect} -f mt9.out -t usb.in0}
\samplecomment{Remove the direct connection between the image sensor and usb communication.}

\subsubsection{showconnects}
\command{\textbf{gpnode} \textbf{showconnects}}

Print the list of flow connections in the current project.

\sampletitle{> \textbf{gpnode} \textbf{showconnects}\\
connects :\\
  + mt9.out -> usb.in0 (msb)\\
  + process1.magnitude -> usb.in0 (lsb)\\
  + mt9.out -> conv.in (msb)\\
  + conv.out -> usb.in1 (msb)\\
  + mt9.out -> process1.in (lsb)\\
  + mt9.out -> lbp.in (msb)\\
  + lbp.out -> usb.in1 (msb)}

\subsection{clock interconnect}
\subsubsection{setclockdomain}
\command{\textbf{gpnode} \textbf{setclockdomain} -n <domain-name> -v <frequency>}

Define a clock frequency \emph{<frequency>} the the clock domain \emph{<domain-name>}.

\sampletitle{> \textbf{gpnode} \textbf{setclockdomain} -n clk\_proc -v 50M}
\samplecomment{Set the main clock domain to 50MHz.}

\subsubsection{showclockdomain}
\command{\textbf{gpnode} \textbf{showclockdomain}}

Print the list of clock domains in the current project.

\sampletitle{> \textbf{gpnode} \textbf{showclockdomain}\\
domains :\\
  + clk\_proc = 48 MHz}

%\subsection{lists}
%\subsubsection{listavailableio}
%\begin{Verbatim}[frame=single]
%gpnode listavailableio
%\end{Verbatim}
%
%\subsubsection{listio}
%\begin{Verbatim}[frame=single]
%gpnode listio
%\end{Verbatim}
%
%\subsubsection{listavailableprocess}
%\begin{Verbatim}[frame=single]
%gpnode listavailableprocess
%\end{Verbatim}
%
%\subsubsection{listprocess}
%\begin{Verbatim}[frame=single]
%gpnode listprocess
%\end{Verbatim}
%
%\subsubsection{listword}
%\begin{Verbatim}[frame=single]
%gpnode listword
%\end{Verbatim}

\end{document}
