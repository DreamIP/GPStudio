\documentclass[10pt,a4paper]{article}
\usepackage[utf8]{inputenc}
\usepackage[english]{babel}
\usepackage[T1]{fontenc}
\usepackage{amsmath}
\usepackage{amsfonts}
\usepackage{amssymb}
\usepackage{lmodern}
\usepackage{fancyvrb}
\usepackage{tikz}
\usepackage{verbatim}

\newcommand{\version}{\IfFileExists{../../version.txt}
{\input{../../version.txt}}
{\input{../../../version.txt}}
}

\newcommand{\command}[1]{%
\indent \fcolorbox{black}{white}{%
   \begin{minipage}{\dimexpr\textwidth-\parindent\relax}%
      #1
   \end{minipage}%
}
}

\newsavebox{\FVerbBox}
\newenvironment{sample}
{\par \vspace{0.2cm} \begin{lrbox}{\FVerbBox}
\begin{minipage}{\dimexpr\textwidth-\parindent\relax}}
{\end{minipage}
\end{lrbox}
\fcolorbox{black}{lightgray}{\usebox{\FVerbBox}}
\vspace{0.2cm}}

\newenvironment{sampletitle}
{\vspace{0.2cm} \noindent\textbf{Example} :
\begin{sample}}
{\end{sample}}

\newcommand{\samplecomment}[1]{%

\textit{#1}
}

\newcommand{\seealso}[1]{\vspace{0.2cm} \noindent\textbf{See also} :\par #1}

% tikz
\usetikzlibrary{calc}
\usetikzlibrary{arrows}
\usetikzlibrary{shadows}

\tikzset{block/.style={draw, text centered, fill=gray!10,drop shadow}}
\tikzset{connect/.style={draw, line width=1 pt}}

\author{Sebastien CAUX}
\title{gplib doc \version}

\begin{document}
\maketitle
\section{Introduction}
gplib is a command line tool that permits to examine and manage GPStudio library.

\section{Use}
gplib always give you information of library including by the current distribution.

Under linux, you have a completion script to help you to writing commands.

\section{Commands}
\subsection{Lists}
\subsubsection{listboard}
\command{\textbf{gplib} \textbf{listboard}}

Give the list of all available board support package.

\sampletitle{> \textbf{gplib} \textbf{listboard}\\
arrow\_sockit de0nano dreamcam\_c3 stratixcam\_s4}

\subsubsection{listio}
\command{\textbf{gplib} \textbf{listio}}

Give the list of all available ios.

\sampletitle{> \textbf{gplib} \textbf{listio}\\
leds mt9 usb\_cypress\_CY7C68014A}

\subsubsection{listprocess}
\command{\textbf{gplib} \textbf{listprocess}}

Give the list of all available processes.

\sampletitle{> \textbf{gplib} \textbf{listprocess}\\
conv gradienthw histogramhw hog lbp normhw roi slidevm}

\subsubsection{listtoolchain}
\command{\textbf{gplib} \textbf{listtoolchain}}

Give the list of all available tool chains.

\sampletitle{> \textbf{gplib} \textbf{listtoolchain}\\
altera\_quartus hdl}

\end{document}
