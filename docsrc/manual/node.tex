\chapter{Camera Node}
\section{Top level of a node}

\begin{figure}[h!]
\centering
\begin{tikzpicture}[node distance=3cm]
\tikzset{blocstyle/.style={block,rectangle,minimum height=1.5cm,minimum width=2cm}};

% blocks
\node[blocstyle] (bloc1) {image sensor};
\node[blocstyle,right of=bloc1] (bloc2) {block1};
\node[blocstyle,right of=bloc2] (bloc3) {block2};
\node[blocstyle,right of=bloc3] (bloc4) {communication};

% SI
\node[blocstyle,fit=(bloc1) (bloc4),inner xsep=0.5cm,inner ysep=0,yshift=2.5cm] (siwb) {Bus Interconnect};
\node[blocstyle,fit=(bloc1) (bloc4),inner xsep=0.5cm,inner ysep=0,yshift=-2.5cm] (mx) {Flow Interconnect};

% Flow to 
\foreach \s in {1,...,4}
{
    \path[connect,->] (bloc\s.east) -| ([xshift=0.2cm]bloc\s.east |- mx.north);
    \path[connect,<-] (bloc\s.west) -| ([xshift=-0.2cm]bloc\s.west |- mx.north);
    \path[connect,<-,double,double distance=0.5mm] (bloc\s.north) -- (bloc\s.north |- siwb.south);
}
\end{tikzpicture}
\caption{Top level of a node}
\end{figure}

\subsection{Special blocks - Network on chip}

intro noc
is a special block generated by GPS

schema avec FI/PI/CI

\subsubsection{FI - Flow Interconnect}
FI - Flow Interconnect connects and re-assign mapping between flow interface blocks.

All the flow interface of each blocks are connected to FI.

\begin{figure}[h!]
\centering
\begin{tikzpicture}[node distance=3cm]
\tikzset{blocstyle/.style={block,rectangle,minimum height=1.5cm,minimum width=2cm}};

% blocks
\node[blocstyle] (bloc1) {block1};
\node[blocstyle,right of=bloc1] (bloc2) {block2};
\node[blocstyle,right of=bloc2] (bloc3) {block3};

% SI
\node[blocstyle,fit=(bloc1) (bloc3),inner xsep=0.5cm,inner ysep=0,yshift=-2cm] (mx) {Flow Interconnect};

% Flow to 
\foreach \s in {2,...,3}
{
    \path[connect,->] (bloc\s.east) -| ([xshift=0.2cm]bloc\s.east |- mx.north);
    \path[connect,<-] (bloc\s.west) -| ([xshift=-0.2cm]bloc\s.west |- mx.north);
}
\path[connect,->] ([yshift=0.1cm]bloc1.east) -| ([xshift=0.4cm]bloc1.east |- mx.north);
\path[connect,->] ([yshift=-0.1cm]bloc1.east) -| ([xshift=0.2cm]bloc1.east |- mx.north);
\path[connect,<-] (bloc1.west) -| ([xshift=-0.2cm]bloc1.west |- mx.north);
\end{tikzpicture}
\caption{FI connexions to blocks}
\end{figure}

This mapping rules are apply :
\begin{itemize}
\item In case of a direct connexion of a flow out interface to one flow input, interface are hard connected through FI.
\item If two flow out are connected to one input, FI create a dynamically configurable multiplexer to permit flow redirection at runtime. This multiplexer is controlled by an internal register.
\end{itemize}

For the size of flow, FI generator process like that :
\begin{itemize}
\item In case of identical size of flow, data of flow interface are connected bit to bit.
\item If the output is larger than the input, an msb connexion is done by default unless you explicitly specify an lsb connexion. Other bits are losted.
\item And if the output is smaller than the input, an msb connexion is done by default unless you explicitly specify an lsb connexion. Zero bits are added as extra bits.
\end{itemize}

ZOOM + explication mux

This block need a special configuration specified in the node project file. This configuration contains all flow connexions between block. It's automatically added by an HDL Toolchain.

\subsubsection{PI - Parameter Interconnect}
PI - Parameter Interconnect is a special block generated by GPS to read/write dynamic parameters of blocks. It connect all masters and slaves BI interface together and create logic control for selecting address space of blocks.

BUS

This block doesn't need any configuration. It's automatically added by an HDL Toolchain.

\subsubsection{CI - Clock Interconnect}
CI or Clock Interconnect is a special block generated by GPS to manage all clocks in the node. It distribute the requested clocks frequency to others blocks taking care of clock domains and clock shifting.

It contains PLL if you need frequency that board can't provide.

This block need a special configuration specified in the node project file. This configuration contains frequency for-each clock domain. It's automatically added by an HDL Toolchain.

%\subsubsection{RI - Reset Interconnect}
%soon available (v0.95)

\section{Properties}
Properties is a way to link block low level hardware registers to high level software controller.
