\chapter{Node}
\section{Top level of a node}

\begin{figure}[h!]
\centering
\begin{tikzpicture}[node distance=3cm]
\tikzset{blocstyle/.style={block,rectangle,minimum height=1.5cm,minimum width=2cm}};

% blocks
\node[blocstyle] (bloc1) {image sensor};
\node[blocstyle,right of=bloc1] (bloc2) {block1};
\node[blocstyle,right of=bloc2] (bloc3) {block2};
\node[blocstyle,right of=bloc3] (bloc4) {communication};

% SI
\node[blocstyle,fit=(bloc1) (bloc4),inner xsep=0.5cm,inner ysep=0,yshift=2.5cm] (siwb) {Bus Interconnect};
\node[blocstyle,fit=(bloc1) (bloc4),inner xsep=0.5cm,inner ysep=0,yshift=-2.5cm] (mx) {Flow Interconnect};

% Flow to 
\foreach \s in {1,...,4}
{
    \path[connect,->] (bloc\s.east) -| ([xshift=0.2cm]bloc\s.east |- mx.north);
    \path[connect,<-] (bloc\s.west) -| ([xshift=-0.2cm]bloc\s.west |- mx.north);
    \path[connect,<-,double,double distance=0.5mm] (bloc\s.north) -- (bloc\s.north |- siwb.south);
}
\end{tikzpicture}
\caption{Top level of a node}
\end{figure}

\section{Properties}
Properties is a way to link block low level hardware registers to high level software controller.
