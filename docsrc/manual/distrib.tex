\chapter{Distribution}

\section{Files organisation}
\begin{figure}[h]
\dirtree{%
.1 GPS distribution root.
   .2 \bfseries bin/.
        .3 gpnode.
        .3 gpviewer.
   .2 \bfseries doc/.
   .2 \bfseries script/.
   .2 \bfseries support/.
        .3 board/.
           .4 board1/.
                .5 board1.dev.
        .3 io/.
        .3 process/.
        .3 toolchain/.
        .3 hwlib/.
}
\caption{Files tree of the main distribution of GPStudio}
\label{fig:archivetree}
\end{figure}

\section{Files format}
All files used by GPStudio are under XML format with specific extension.

\subsection{.dev}
This file is the definition of a board platform, his capabilities and functionalities.

It contains the toolchain to produce a byte code, special attributes for supporting the board and the list of available peripheral. For each peripheral, 

\subsection{.proc}
files implementation list with type of file and path

\subsection{.io}
Io file definition is quite similar of process file with addition of an external interface.

\subsection{.node}

\section{Tools}
\subsection{gpnode}
gpnode is a command line tool that permit to manage a gptudio node project.

\emph{See also gpnode manual.}

\subsection{gplib}

\subsection{gpproc}

\subsection{gpdevice}

\subsection{gpviewer}
gpviewer is a graphical interface to see results of processing and setting up your smart camera during execution.

\emph{See also gpviewer manual.}

\section{Work flow}
